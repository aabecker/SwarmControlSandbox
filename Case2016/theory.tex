%%%%%%%%%%%%%%%%%%%%%%%%%%%%%%%%%%%%%%%%%%%%%%%%%%%%%%%%%%%
\section{Torque Control}
\label{sec:theory}
%%%%%%%%%%%%%%%%%%%%%%%%%%%%%%%%%%%%%%%%%%%%%%%%%%%%%%%%%%%
%\subsection{Controlling Torque}


The orientation of an object's major axis is important when a swarm is manipulating a non-symmetric object through narrow corridors. 
Orientation is controllable by applying torque to the object. 
To change the output torque $\tau$ in Eq.~\eqref{eq:torque}, we can choose the direction and magnitude of the force applied $F$, and the moment arm from the object's center of mass (COM) to the point of contact $r$.

\begin{equation}
\tau = F \times r\label{eq:torque}
\end{equation}
The swarm version of \eqref{eq:torque} is the summation of the forces contributed by individual robots.

\begin{align}
\tau_{total} &= \sum\limits_{i=1}^n \rho_i F_i \times (P_i - O )   \label{eq:swarmtorque}\\
F_{total} &= \sum\limits_{i=1}^n \rho_i F_i  \label{eq:swarmforce}
\end{align}

Here $F_i$ is the force that the $i$th robot applies.  If all robots are identical and the control input is uniform, the force is equivalent for every robot and $F_i = F_c$.
Not all robots are in contact with the object.  $\rho_i$ is an indicator variable: $\rho_i$ is 1 if the robot is in direct contact with the object or touching a chain of robots where at least one robot is in contact with the object. Otherwise $\rho_i = 0$.
The moment arm is the robot's position $P_i$ to the object's COM $O=[O_x,O_y]^{\top}$.




